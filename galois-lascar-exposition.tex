\documentclass[letterpaper,twoside]{article}
\usepackage{amsmath}
\usepackage{amssymb}
\usepackage{amsthm}

\newtheorem{theorem}{Theorem}[section]
\newtheorem{proposition}[theorem]{Proposition}
\newtheorem{lemma}{Lemma}[theorem]

\theoremstyle{definition}
\newtheorem{definition}[theorem]{Definition}

\theoremstyle{remark}
\newtheorem{remark}[theorem]{Remark}

\newcommand{\defterm}[1]{\emph{#1}}

\renewcommand{\emptyset}{\varnothing}
\DeclareMathOperator{\tp}{tp}
\DeclareMathOperator{\sym}{sym}

\title{Exposition of \(Gal_L(T)\)}
\author{Mike Haskel}
\date{Summer 2014}

\begin{document}
\maketitle

\section{Introduction}

The purspose of this writeup is to construct \(Gal_L(T)\) without
making any arbitrary choices, such as selecting a large saturated
model.

\section{Bounded, \(Aut\)-Invariant Equivalences}

In this section, I will construct the Lascar equivalence relation in
terms of types.

\begin{definition}
  Let \(T\) be a complete theory.  An \(n\)-ary \defterm{relation
    present in \(T\)} is a subset of \(S_n(T/\emptyset)\).  For any
  \(R\) a relation present in \(T\), \(a_1,\ldots,a_n\) in an
  appropriate context, say that \(R(a_1,\ldots,a_n)\) holds just if
  \(\tp(a_1,\ldots,a_n) \in R\).
\end{definition}

This definition corresponds to the usual definition of an
automorphism-invariant relation.

\begin{proposition}
  Let \(\mathcal{U}\) be a sufficiently large model.  An \(n\)-ary
  relation \(R\) on this model is automorphism-invariant iff it arises
  as a relation present in \(T\).
\end{proposition}
\begin{proof}
  Let \(R\) be a relation present in \(T\), \(f \in
  Aut(\mathcal{U})\), and \(a\in \mathcal{U}^n\).  Since \(f\) is an
  automorphism, \(\tp(a) = \tp(f(a))\).  Therefore \(R(a)\) iff
  \(R(f(a))\), so \(R\) is automorphism-invariant.

  Let \(R\) be an automorphism-invariant relation on
  \(\mathcal{U}^n\), and let \(a,b \in \mathcal{U}^n\) have the same
  type.  Since \(\mathcal{U}\) is sufficiently homogeneous, there is
  an automorphism \(f\) with \(f(a) = b\).  Since \(R\) is
  automorphism-invariant, \(R\) agrees on \(a\) and \(b\).  Therefore
  whether \(R\) holds of a tuple depends only on the tuple's type, so
  \(R\) arises as a type present in \(T\).
\end{proof}

We can define reflexivity, symmetry, and transitivity of a binary
relation present in \(T\) without referencing any models.

\begin{definition}
  Let \(R(x,y)\) be a binary relation present in \(T\).  \(R\) is
  \defterm{reflexive} if \([x = y] \subseteq R\), where \([x = y]\) is
  the filter generated by the given formula, viewed as a (closed) set
  of types.
\end{definition}

\begin{proposition}
  Let \(R(x,y)\) be a binary relation present in \(T\).  \(R\) is
  reflexive iff it yields a reflexive relation in all models, iff it
  yields a reflexive relation in a sufficiently large model.
\end{proposition}
\begin{proof}
  Let \(R(x,y)\) be a reflexive relation present in \(T\), \(M \models
  T\), and \(a \in M\).  Since \(\tp(a,a) \in [x = y] \subseteq R\),
  \(R(a,a)\).  Therefore \(R\) yields a reflexive relation in all models.

  Let \(R(x,y)\) be a relation present in \(T\) which is not
  reflexive.  Let \(\mathcal{U} \models T\) be sufficiently large.
  Since \(R\) is not reflexive, take \(p(x,y) \in [x = y]\) with \(p
  \notin R\).  Since \(\mathcal{U}\) is sufficiently saturated, take
  \(a,b \in \mathcal{U}\) with \(\mathcal{U} \models p(a,b)\).  Since
  \(p \in [x = y]\), \(a = b\).  However, since \(p \notin R\), \(\neg
  R(a,b)\).  Therefore \(R\) does not yield a reflexive relation in
  \(\mathcal{U}\).
\end{proof}

\begin{definition}
  Let \(R(x,y)\) be a binary relation present in \(T\).  \(R\) is
  \defterm{symmetric} if, for all \(p(x,y) \in R\), \(p(y,x) \in R\).
\end{definition}

\begin{proposition}
  Let \(R(x,y)\) be a binary relation present in \(T\).  \(R\) is
  symmetric iff it yields a symmetric relation in all models, iff it
  yields a symmetric relation in a sufficiently large model.
\end{proposition}
\begin{proof}
  Let \(R(x,y)\) be a symmetric relation present in \(T\), \(M \models
  T\), and \(a,b \in M\) with \(R(a,b)\).  Set \(p(x,y) = \tp(a,b)\).
  Since \(\tp(b,a) = p(y,x) \in R\), \(R(b,a)\).  Therefore \(R\)
  yields a symmetric relation in \(M\).

  Let \(R(x,y)\) be a binary relation present in \(T\) which is not
  symmetric.  Let \(\mathcal{U}\) be a sufficiently large model.
  Since \(R\) is not symmetric, take \(p(x,y) \in R\) with \(p(y,x)
  \notin R\).  Since \(\mathcal{U}\) is sufficiently saturated, take
  \(a,b \in \mathcal{U}\) with \(\tp(a,b) = p(x,y)\), so \(R(a,b)\).
  However, \(\tp(b,a) = p(y,x) \notin R\), so \(\neg R(b,a)\).
  Therefore \(R\) does not yield a symmetric relation in
  \(\mathcal{U}\).
\end{proof}

\begin{definition}
  Let \(R(x,y)\) be a binary relation present in \(T\).  \(R\) is
  \defterm{transitive} if, for all \(p(x,y), q(x,y) \in R\),
  \([p(x,z), q(z,y)]\restriction_{x,y} \subseteq R\), where \([p(x,z),
    q(z,y)]\restriction_{x,y}\) is the filter in \(x,y\) obtained by
  restricting the filter in \(x,y,z\) generated by the formulas in
  \(p(x,z)\) and \(q(z,y)\) to those formulas only mentioning \(x\)
  and \(y\).
\end{definition}

\begin{proposition}
  Let \(R(x,y)\) be a binary relation present in \(T\).  \(R\) is
  transitive iff it yields a transitive relation in all models, iff it
  yields a transitive relation in a sufficiently large model.
\end{proposition}
\begin{proof}
  Let \(R(x,y)\) be a transitive relation present in \(T\), \(M
  \models T\), \(a,b,c \in M\) with \(R(a,b)\) and \(R(b,c)\).  Set
  \(p(x,y) = \tp(a,b)\) and \(q(x,y) = \tp(b,c)\).  We know that \(p,q
  \in R\).  Furthermore, \(r(x,z,y) := \tp(a,b,c) \in [p(x,z),
    q(z,y)]\), so \(\tp(a,c) = r\restriction_{x,y} \in R\), and
  therefore \(R(a,c)\).  Therefore \(R\) yields a transitive relation
  in \(M\).

  Let \(R(x,y)\) be a binary relation present in \(T\) which is not
  transitive.  Let \(\mathcal{U}\) be a sufficiently large model.
  Since \(R\) is not transitive, take \(p(x,y), q(x,y) \in R\),
  \(r(x,z,y) \in [p(x,z), q(z,y)]\), with \(r\restriction_{x,y} \notin
  R\).  Since \(\mathcal{U}\) is sufficiently saturated, take \(a,b,c
  \in \mathcal{U}\) with \(\tp(a,b,c) = r\).  This implies that
  \(\tp(a,b) = p\) and \(\tp(b,c) = q\), so \(R(a,b)\) and \(R(b,c)\).
  However, since \(\tp(a,c) = r\restriction_{x,y}\), \(\tp(a,c) \notin
  R\) and \(\neg R(a,c)\).  Therefore \(R\) does not yield a
  transitive relation in \(\mathcal{U}\).
\end{proof}

For an equivalence relation present in \(T\), there is a purely
combinatorial definition of boundedness.  We want to say that an
equivalence relation is bounded if there aren't too many equivalence
classes.  Since an equivalence relation present in \(T\) is just a set
of types over the empty set, the ways in which elements can be in
different equivalence classes correspond to those types \emph{not} in
the relation.  By Ramsey theory, if there are too many inequivalent
elements, there are infinitely many elements inequivalent by the same
type.  By compactness, it is therefore possible to have arbitrarily
many elements inequivalent by this type.  This characterization is
formalized below.

\begin{definition}
  Let \(p(x,y) \in S_2(T)\).  Then \(\sym(p)\), the \defterm{symmetric
    closure} of \(p\), is \(\{p(x,y), p(y,x)\}\).  Since \(\sym(p)
  \subseteq S_2(T)\), \(\sym(p)\) is a (symmetric) relation present in
  \(T\).  Additionally, since \(\sym(p)\) is a finite set of points,
  it is closed, and therefore corresponds to a filter of formulas.
\end{definition}

\begin{definition}
  Let \(p(x,y) \in S_2(T)\).  \(\sym(p)\) is \defterm{graph-infinite}
  if \[[\sym(p)(x_i,x_j), x_i \neq x_j : i \neq j],\] a filter in the
  variables \(\{x_i\}_{i \in \omega}\), is consistent.
\end{definition}

\begin{proposition}\label{proposition-graphinfinite}
  Let \(p(x,y) \in S_2(T)\).  Then the following are equivalent:
  \begin{itemize}
  \item \(\sym(p)\) is graph-infinite;
  \item for some infinite \(\kappa\), there is a model with \(\kappa\)
    many elements pairwise related by \(\sym(p)\);
  \item for all infinite \(\kappa\), there is a model with \(\kappa\)
    many elements pairwise related by \(\sym(p)\); and
  \item for any \(\kappa\)-saturated model \(\mathcal{U}\), there are
    at least \(\kappa\) many elements of \(\mathcal{U}\) pairwise
    related by \(\sym(p)\).
  \end{itemize}
\end{proposition}
\begin{proof}
  This is an easy consequence of compactness.
\end{proof}

\begin{definition}
  Let \(xEy\) be an equivalence relation present in \(T\).  \(E\) is
  \defterm{bounded} if no \(p(x,y) \notin E\) is graph-infinite.
\end{definition}

This definition aligns with the traditional definition of boundedness.

\begin{proposition}
  Let \(xEy\) be an equivalence relation present in \(T\).  \(E\) is
  bounded iff, in a sufficiently large model \(\mathcal{U}\), the
  number of equivalence classes is strictly less than \(\kappa\),
  where \(\kappa\) is the saturation of \(\mathcal{U}\).
\end{proposition}
\begin{proof}
  (\(\Leftarrow\)) Assume that \(E\) is not bounded.  Then take
  \(p(x,y) \notin E\) graph-infinite.  By
  Proposition~\ref{proposition-graphinfinite}, there are \(\kappa\)
  many elements of \(\mathcal{U}\) pairwise related by \(\sym(p)\).
  Since \(p \notin E\), and since \(E\) is symmetric, \(\sym(p)\) is
  disjoint from \(E\).  Therefore we have found \(\kappa\) many
  pairwise inequivalent elements of \(\mathcal{U}\).

  (\(\Rightarrow\)) Let \(\tau = |S_2(T)|\).  By Ramsey theory, there
  is a least cardinal \(\lambda\) such that, for any coloring of the
  complete 2-graph with \(\lambda\) vertices and \(\tau\) colors,
  there is a complete, infinite, homogeneous subgraph.  Since
  \(\mathcal{U}\) is sufficiently large, in particular \(\kappa \geq
  \lambda\).

  Assume that there are at least \(\kappa\) equivalence classes in
  \(\mathcal{U}\).  Pick one representative from each of the \(\geq
  \kappa\) classes.  Consider each pair of representatives \(\{a,b\}\)
  ``colored'' by \(\sym(a,b) = \sym(b,a)\).  Since \(\kappa \geq
  \lambda\), there is an complete, infinite, homogeneous subgraph.
  That is, there is some graph-infinite \(\sym(p)\) with \(p \notin
  E\).  Therefore \(E\) is not bounded.
\end{proof}

The definition of the Lascar equivalence relation, \(E_L\), of a
theory proceeds as usual.

\begin{definition}\label{definition-lascarequivalence}
  Let \(T\) be a complete theory.  The \defterm{Lascar equivalence
    relation}, \(E_L\), for that theory is the intersection (as sets
  of types) of all bounded equivalence relations.
\end{definition}

\begin{remark}\label{remark-relationintersection}
  It is clear that intersection of relations as sets of types
  corresponds with the usual intersection of relations in any model.
\end{remark}

\begin{proposition}
  For any theory, \(E_L\) is a bounded equivalence relation.
\end{proposition}
\begin{proof}
  From Remark~\ref{remark-relationintersection}, since a relation
  present in a theory is an equivalence relation iff it yields an
  equivalence relation in a sufficiently saturated model, and since
  the intersection of (usual) equivalence relations is an equivalence
  relation, \(E_L\) is an equivalence relation.  It remains to show
  that \(E_L\) is bounded.

  Assume that \(E_L\) is not bounded.  Then there is some \(p \notin
  E_L\) such that \(\sym(p)\) is graph-infinite.  Since \(E_L\) is the
  intersection of bounded equivalence relations, \(p \notin E\) for
  some bounded equivalence relation \(E\).  However, this contradicts
  \(\sym(p)\) being graph-infinite.
\end{proof}

Finally, this is all just an alternate exposition of the usual
definition.

\begin{theorem}
  The Lascar equivalence relation, as defined in
  Definition~\ref{definition-lascarequivalence}, induces the usual
  Lascar equivalence relation in any sufficiently large model
  \(\mathcal{U}\).
\end{theorem}
\begin{proof}
  This is a direct combination of the preceding results.
\end{proof}

\end{document}
