\documentclass[letterpaper,twoside]{article}
\usepackage{amsmath}
\usepackage{amssymb}
\usepackage{amsthm}

\newtheorem{theorem}{Theorem}[section]
\newtheorem{proposition}[theorem]{Proposition}
\newtheorem{lemma}{Lemma}[theorem]

\theoremstyle{definition}
\newtheorem{definition}[theorem]{Definition}

\theoremstyle{remark}
\newtheorem{remark}[theorem]{Remark}

\newcommand{\defterm}[1]{\emph{#1}}

\renewcommand{\emptyset}{\varnothing}
\DeclareMathOperator{\tp}{tp}

\title{Exposition of \(Gal_L(T)\)}
\author{Mike Haskel}
\date{Summer 2014}

\begin{document}
\maketitle

\section{Introduction}

The purspose of this writeup is to construct \(Gal_L(T)\) without
making any arbitrary choices, such as selecting a large saturated
model.

\section{Bounded, Automorphism-Invariant Equivalences}

In this section, I will construct the Lascar equivalence relation in
terms of types.

\begin{definition}
  Let \(T\) be a complete theory.  An \(n\)-ary \defterm{relation
    present in \(T\)} is a subset of \(S_n(T/\emptyset)\).  For any
  \(R\) a relation present in \(T\), \(a_1,\ldots,a_n\) in an
  appropriate context, say that \(R(a_1,\ldots,a_n)\) holds just if
  \(\tp(a_1,\ldots,a_n) \in R\).
\end{definition}

This definition corresponds to the usual definition of an
automorphism-invariant relation.

\begin{proposition}
  Let \(\mathcal{U}\) be a sufficiently large model.  An \(n\)-ary
  relation \(R\) on this model is automorphism-invariant iff it arises
  as a relation present in \(T\).
\end{proposition}
\begin{proof}
  Let \(R\) be a relation present in \(T\), \(f \in
  Aut(\mathcal{U})\), and \(a\in \mathcal{U}^n\).  Since \(f\) is an
  automorphism, \(\tp(a) = \tp(f(a))\).  Therefore \(R(a)\) iff
  \(R(f(a))\), so \(R\) is automorphism-invariant.

  Let \(R\) be an automorphism-invariant relation on
  \(\mathcal{U}^n\), and let \(a,b \in \mathcal{U}^n\) have the same
  type.  Since \(\mathcal{U}\) is sufficiently homogeneous, there is
  an automorphism \(f\) with \(f(a) = b\).  Since \(R\) is
  automorphism-invariant, \(R\) agrees on \(a\) and \(b\).  Therefore
  whether \(R\) holds of a tuple depends only on the tuple's type, so
  \(R\) arises as a type present in \(T\).
\end{proof}

We can define reflexivity, symmetry, and transitivity of a binary
relation present in \(T\) without referencing any models.

\begin{definition}
  Let \(R(x,y)\) be a binary relation present in \(T\).  \(R\) is
  \defterm{reflexive} if \([x = y] \subseteq R\), where \([x = y]\) is
  the filter generated by the given formula, viewed as a (closed) set
  of types.
\end{definition}

\begin{proposition}
  Let \(R(x,y)\) be a binary relation present in \(T\).  \(R\) is
  reflexive iff it yields a reflexive relation in all models, iff it
  yields a reflexive relation in a sufficiently large model.
\end{proposition}
\begin{proof}
  Let \(R(x,y)\) be a reflexive relation present in \(T\), \(M \models
  T\), and \(a \in M\).  Since \(\tp(a,a) \in [x = y] \subseteq R\),
  \(R(a,a)\).  Therefore \(R\) yields a reflexive relation in all models.

  Let \(R(x,y)\) be a relation present in \(T\) which is not
  reflexive.  Let \(\mathcal{U} \models T\) be sufficiently large.
  Since \(R\) is not reflexive, take \(p(x,y) \in [x = y]\) with \(p
  \notin R\).  Since \(\mathcal{U}\) is sufficiently saturated, take
  \(a,b \in \mathcal{U}\) with \(\mathcal{U} \models p(a,b)\).  Since
  \(p \in [x = y]\), \(a = b\).  However, since \(p \notin R\), \(\neg
  R(a,b)\).  Therefore \(R\) does not yield a reflexive relation in
  \(\mathcal{U}\).
\end{proof}

\begin{definition}
  Let \(R(x,y)\) be a binary relation present in \(T\).  \(R\) is
  \defterm{symmetric} if, for all \(p(x,y) \in R\), \(p(y,x) \in R\).
\end{definition}

\begin{proposition}
  Let \(R(x,y)\) be a binary relation present in \(T\).  \(R\) is
  symmetric iff it yields a symmetric relation in all models, iff it
  yields a symmetric relation in a sufficiently large model.
\end{proposition}
\begin{proof}
  Let \(R(x,y)\) be a symmetric relation present in \(T\), \(M \models
  T\), and \(a,b \in M\) with \(R(a,b)\).  Set \(p(x,y) = \tp(a,b)\).
  Since \(\tp(b,a) = p(y,x) \in R\), \(R(b,a)\).  Therefore \(R\)
  yields a symmetric relation in \(M\).

  Let \(R(x,y)\) be a binary relation present in \(T\) which is not
  symmetric.  Let \(\mathcal{U}\) be a sufficiently large model.
  Since \(R\) is not symmetric, take \(p(x,y) \in R\) with \(p(y,x)
  \notin R\).  Since \(\mathcal{U}\) is sufficiently saturated, take
  \(a,b \in \mathcal{U}\) with \(\tp(a,b) = p(x,y)\), so \(R(a,b)\).
  However, \(\tp(b,a) = p(y,x) \notin R\), so \(\neg R(b,a)\).
  Therefore \(R\) does not yield a symmetric relation in
  \(\mathcal{U}\).
\end{proof}

\end{document}
